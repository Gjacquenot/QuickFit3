%% Based on a TeXnicCenter-Template by Gyorgy SZEIDL.
%%%%%%%%%%%%%%%%%%%%%%%%%%%%%%%%%%%%%%%%%%%%%%%%%%%%%%%%%%%%%

%------------------------------------------------------------
%
\documentclass[a4paper,notitlepage]{article}
%
%----------------------------------------------------------
% This is a sample document for the AMS LaTeX Article Class
% Class options
%        -- Point size:  8pt, 9pt, 10pt (default), 11pt, 12pt
%        -- Paper size:  letterpaper(default), a4paper
%        -- Orientation: portrait(default), landscape
%        -- Print size:  oneside, twoside(default)
%        -- Quality:     final(default), draft
%        -- Title page:  notitlepage, titlepage(default)
%        -- Start chapter on left:
%                        openright(default), openany
%        -- Columns:     onecolumn(default), twocolumn
%        -- Omit extra math features:
%                        nomath
%        -- AMSfonts:    noamsfonts
%        -- PSAMSFonts  (fewer AMSfonts sizes):
%                        psamsfonts
%        -- Equation numbering:
%                        leqno(default), reqno (equation numbers are on the right side)
%        -- Equation centering:
%                        centertags(default), tbtags
%        -- Displayed equations (centered is the default):
%                        fleqn (equations start at the same distance from the right side)
%        -- Electronic journal:
%                        e-only
%------------------------------------------------------------
% For instance the command
%          \documentclass[a4paper,12pt,reqno]{amsart}
% ensures that the paper size is a4, fonts are typeset at the size 12p
% and the equation numbers are on the right side
%
\usepackage{amsmath}%
\usepackage{amsfonts}%
\usepackage{amssymb}%
\usepackage{graphicx}
\usepackage{hyperref}
\usepackage{doi}
\usepackage[numbers]{natbib}
\usepackage{dsfont}
\usepackage[left=2cm,right=1.5cm,top=1.5cm,bottom=1.5cm]{geometry}


\DeclareMathOperator*{\argmin}{arg\:min\ }
\DeclareMathOperator*{\argmax}{arg\:max\ }
\DeclareMathOperator{\Var}{Var}
\newcommand{\mat}[1]{\mathrm{\mathbf{#1}}}

\begin{document}
\title{Bayesian Model Selection (for FCS)}
\author{Jan Krieger}
\date{\today}
\maketitle
\tableofcontents

\section{Standard FCS data-fitting}
\subsection{Least-Squares Fit}
In standard FCS-Fitting the following least-squares problem is solved:
\begin{equation}\label{eq:fcs_fit_problem}
  \vec{\beta}_{M_g}^\ast=\argmin\limits_{\vec{\beta}}\underbrace{\sum\limits_{i=1}^n\left|\frac{g(\hat{\tau}_i;\vec{\beta})-\hat{g}_i}{\hat{\sigma}_i}\right|^2}_{=:\chi^2(\vec{\beta})},
\end{equation}
where $g(\tau;\vec{\beta})$ is the fit model (later also denoted as $M_g$ to enumerate different models) and $\vec{\beta}\in\mathbb{R}^k$ are the $k$ parameters of the model function. The measurement consists of $n$ value-pairs $(\hat{\tau}_i,\hat{g}_i)$ and errors for each datapoint $\hat{\sigma}_i$ that are used to weight the fits.

\subsection{Least-Squares Fit Algorithms (Basics)}
The least-squares fit problem \eqref{eq:fcs_fit_problem} can be formulated in a more compact form by writing the objective function $\chi^2(\vec{p})$ in the following way:
\begin{equation}\label{eq:fcs_fit_problem_forfit}
  \vec{\beta}_{M_g}^\ast=\argmin\limits_{\vec{\beta}}\chi^2(\vec{\beta})%=\\
  =\argmin\limits_{\vec{\beta}}\left\|\vec{F}(\vec{\beta})\right\|_2^2=\argmin\limits_{\vec{\beta}}\sum\limits_{i=1}^n\bigl[f_i(\beta_1,\beta_2,...,\beta_k)\bigr]^2
\end{equation}

Most numerical fit algorithms, that solve this problem, start from an initial guess $\vec{\beta}_0$ and then find the best-fit parameters by proceeding along the steepest descent of $\chi^2$. This is done in an iterative way by solving this linearized problem in each step (Gauss-Newton iteration, probably with additional conditioning, such as in the LM-fit):
\begin{equation}\label{eq:fcs_fit_problem_linearization}
  \vec{\beta}_{i+1}=\vec{\beta}_{i}+\vec{\delta}_{i+1}=\vec{\beta}_{i}+\argmin\limits_{\vec{\delta}}\left\|\vec{F}(\vec{\beta}_i)+\mat{J}(\vec{\beta}_i)\vec{\delta}\right\|_2^2.
\end{equation}
Here $\mat{J}(\vec{\beta})$ is the jacobi matrix, i.e. the matrix of first derivatives:
\begin{equation}\label{eq:fit_jacobi}
    J_{\nu,\kappa}(\vec{\beta})=\left.\frac{\partial f_\nu}{\partial p_\kappa}\right|_{\vec{\beta}}=\frac{1}{\hat{\sigma}_\nu}\cdot\left.\frac{\partial g(\hat{\tau}_\nu;\vec{\beta})}{\partial p_\kappa}\right|_{\vec{\beta}}
\end{equation}

The problem \eqref{eq:fcs_fit_problem_linearization} is a linear system of equations, which can be solved as follows by normal equations, which follow from requiring that the gradient of $\chi^2$ equals zero:
\begin{align}
  \vec{F}(\vec{\beta}_i)+\mat{J}(\vec{\beta}_i)\vec{\delta}&\overset{!}{=}0&&\Leftrightarrow&-\mat{J}(\vec{\beta}_i)^\mathrm{T}\vec{F}(\vec{\beta}_i)&\overset{!}{=}\left[\mat{J}(\vec{\beta}_i)^\mathrm{T}\mat{J}(\vec{\beta}_i)\right]\vec{\delta}\notag\\
  &&&\Leftrightarrow&\vec{\delta}&=-\left[\mat{J}(\vec{\beta}_i)^\mathrm{T}\mat{J}(\vec{\beta}_i)\right]^{-1}\mat{J}(\vec{\beta}_i)^\mathrm{T}\vec{F}(\vec{\beta}_i)\label{eq:fcs_fit_problem_linearization_solution}
\end{align}

\subsection{Fit parameter errors/variance-covariance matrix}
\label{sec:varcovmatrix}
The next question that arises is, how accurate do we know the parameters in the best-fit parameter vector $\vec{\beta}^\ast$? 
To solve this problem, we look again at the least squares problem \eqref{eq:fcs_fit_problem} and write it in terms of data vectors $\hat{\vec{g}}=[\hat{g}_1,\hat{g}_2,...]^\mathrm{T}$ and a vector-valued fit function $\vec{g}(\vec{\beta})=[g(\hat{\tau}_1;\vec{\beta}), g(\hat{\tau}_2;\vec{\beta}),...]^\mathrm{T}$. If we omit the weights $\hat{\sigma}_i$, we can write for the ideal case of a perfect fit:
\begin{equation}\label{eq:varcov_dervation1}
   \hat{\vec{g}}=\vec{g}(\vec{\beta}) 
\end{equation}
Now we have small changes $\vec{\epsilon}$ of the data around the ideal values $\hat{\vec{g}}$. Since these fluctuations are small, it should be possible to account for them by a first-order Taylor approximation of the fit function $\vec{g}(\vec{\beta})$ and therefore a small (linear) variation of the best fit parameters $\vec{\beta}$:
\begin{equation}\label{eq:varcov_dervation2}
   \hat{\vec{g}}+\vec{\epsilon}=\vec{g}(\vec{\beta})+\mat{J}\delta\vec{\beta} 
\end{equation}
Using \eqref{eq:varcov_dervation1} this can be rewritten and solved for $\delta\vec{\beta}$ with the same method as in \eqref{eq:fcs_fit_problem_linearization_solution} (only now written in a short form):
\begin{equation}\label{eq:varcov_dervation3}
   \vec{\epsilon}=\mat{J}\delta\vec{\beta} \ \ \ \ \ \Rightarrow\ \ \ \ \ \delta\vec{\beta}=\left[\mat{J}^\mathrm{T}\mat{J}\right]^{-1}\mat{J}^\mathrm{T}\vec{\epsilon}\equiv \mat{C}\vec{\epsilon}
\end{equation}

  
These findings can now be used to calculate an approximation for the variance of the ideal parameters:
\begin{align}\label{eq:varcov_dervation4}
    \Var(\vec{\beta})&=\Var(\delta\vec{\beta})=\Var\left(\mat{C}\vec{\epsilon}\right)=\mat{C}^\mathrm{T}\Var(\vec{\epsilon})\mat{C}=\notag\\
    &=\frac{\chi^2}{n-k}\cdot\left(\left[\mat{J}^\mathrm{T}\mat{J}\right]^{-1}\mat{J}^\mathrm{T}\right)\left(\mat{J}\left[\mat{J}^\mathrm{T}\mat{J}\right]^{-1}\right)=\notag\\
    &=\frac{\chi^2}{n-k}\cdot\left[\mat{J}^\mathrm{T}\mat{J}\right]^{-1}\underbrace{\mat{J}^\mathrm{T}\mat{J}\left[\mat{J}^\mathrm{T}\mat{J}\right]^{-1}}_{=\mathds{1}}=\frac{\chi^2}{n-k}\cdot\left[\mat{J}^\mathrm{T}\mat{J}\right]^{-1}
\end{align}
From the first to the second line we estimate the variance of the data variation as the variance of the residulas $\Var(\vec{\epsilon})=(\chi^2/(n-k))\cdot \mathds{1}$. 

So finally we can define the covariance matrix of the non-linear least-squares problem as:
\begin{equation}\label{eq:varcov_noerrors}
   \mat{\Sigma}=\left[\mat{J}^\mathrm{T}\mat{J}\right]^{-1}
\end{equation}
and the standard error of a parameter as
\begin{equation}\label{eq:stderr_noerrors}
   \mbox{err}(\beta_i)=\sqrt{\frac{\chi^2}{n-k}\cdot\mat{\Sigma}_{ii}}.
\end{equation}








\subsection{Likelihood function}
The likelihood function for the problem \eqref{eq:fcs_fit_problem} can be written exactly, if we assume independent errors $\hat{\sigma}_i$ for each measured point $(\hat{\tau}_i, \hat{g}_i)$ on the ACF and a Gaussian error distribution:
\begin{equation}\label{eq:fcs_gauss_errors}
  p(\hat{g}_i|M_g,\vec{\beta})=\frac{1}{\sqrt{2\pi}\cdot\hat{\sigma}_i}\cdot\exp\left[-\frac{1}{2}\cdot\frac{\left(g(\hat{\tau}_i;\vec{\beta})-\hat{g}_i\right)^2}{\hat{\sigma}_i^2}\right].
\end{equation}
This $p(\hat{\tau}_i,\hat{g}_i|M_g,\vec{\beta})$ is the conditional probability to measure the value $\hat{g}_i$ of the ACF at the (given) lag-time $\hat{\tau}_i$, given the specific FCS model function $M_g$ and the parameter vector $\vec{\beta}$. Equation \eqref{eq:fcs_gauss_errors} can then be used to derive the likelihood function for this problem, which is basically the net probability to obtain the complete set $i=1..n$ of measurements, given again the model $M_g$ and the parameter vector $\vec{\beta}$:
\begin{equation}\label{eq:fcs_likelihood_simple}
  p(\hat{\vec{g}}|M_g,\vec{\beta})=\prod\limits_{i=1}^np(\hat{g}_i|M_g,\vec{\beta})%=\\
  =\frac{1}{(2\pi)^{n/2}}\cdot\frac{1}{\prod\limits_i\hat{\sigma}_i}\cdot\exp\left[-\sum\limits_{i=1}^n\frac{1}{2}\cdot\frac{\left(g(\hat{\tau}_i;\vec{\beta})-\hat{g}_i\right)^2}{\hat{\sigma}_i^2}\right].
\end{equation}
If the errors are not independent, then they are no longer described by the $\hat{\sigma}_i$, but by an $n\times n$ covariance matrix $\hat{\mat{C}}$ that has to be determined from the measurement, or from theoretical considerations. The likelihood is then given by:
\begin{equation}\label{eq:fcs_likelihood_covmatrix}
  p(\hat{\vec{g}}|M_g,\vec{\beta})=\prod\limits_{i=1}^np(\hat{g}_i|M_g,\vec{\beta})%=\\
  =\frac{1}{(2\pi)^{n/2}}\cdot\frac{1}{\sqrt{\det(\hat{\mat{C}})}}\cdot\exp\left[-\frac{1}{2}\cdot\bigl[\hat{\vec{g}}-\vec{g}(\hat{\vec{\tau}}; \vec{\beta})\bigr]^\mathrm{T}\hat{\mat{C}}^{-1} \bigl[\hat{\vec{g}}-\vec{g}(\hat{\vec{\tau}}, \hat{\vec{g}}; \vec{\beta})\bigr]\right].
\end{equation}
Note: Here the single measurements $i$ have been combined into vector, e.g. $\hat{\vec{g}}=\left[\hat{g}_1, \hat{g}_2, ..., \hat{g}_n\right]^\mathrm{T}$. The model is then also vector-valued with $\vec{g}(\hat{\vec{\tau}}; \vec{\beta})=\left[g(\hat{\tau}_1;\vec{\beta}), g(\hat{\tau}_2;\vec{\beta}), ..., g(\hat{\tau}_n;\vec{\beta})\right]^\mathrm{T}$

Then the least-squares problem \eqref{eq:fcs_fit_problem} can also be written as a maximum-likelihood estimate (MLE):
\begin{equation}\label{eq:fcs_fit_problem_maxlikelihood}
  \vec{\beta}_{M_g}^\ast=\argmax\limits_{\vec{\beta}}p(\hat{\vec{g}}|M_g,\vec{\beta}).
\end{equation}
Introducing the log-likelihood $\ln \bigl[p(\hat{\vec{g}}|M_g,\vec{\beta})\bigr]$, this can be rewritten in terms of sums, not products and allows to remove the exponential functions:
\begin{align*}
  \vec{\beta}_{M_g}^\ast&=\argmax\limits_{\vec{\beta}}p(\hat{\vec{g}}|M_g,\vec{\beta})=\argmax\limits_{\vec{\beta}}\ln\bigl[p(\hat{\vec{g}}|M_g,\vec{\beta})\bigr]=\\
    &=\argmax\limits_{\vec{\beta}}\left\{-\frac{n}{2}\cdot\ln(2\pi)-\sum\limits_i\ln\left[\hat{\sigma}_i\right]-\frac{1}{2}\cdot\sum\limits_{i=1}^n\frac{\left(g(\hat{\tau}_i;\vec{\beta})-\hat{g}_i\right)^2}{\hat{\sigma}_i^2}\right\}=\\
    &=\argmin\limits_{\vec{\beta}}\left\{\frac{n}{2}\cdot\ln(2\pi)+\sum\limits_i\ln\left[\hat{\sigma}_i\right]+\frac{1}{2}\cdot\sum\limits_{i=1}^n\frac{\left(g(\hat{\tau}_i;\vec{\beta})-\hat{g}_i\right)^2}{\hat{\sigma}_i^2}\right\}=\\
    &=\argmin\limits_{\vec{\beta}} \sum\limits_{i=1}^n\frac{\left(g(\hat{\tau}_i;\vec{\beta})-\hat{g}_i\right)^2}{\hat{\sigma}_i^2}
\end{align*}
In the last line, constant term and factors (that do not depend on $\vec{\beta}$) were omitted and the problem was reduced again to \eqref{eq:fcs_fit_problem}, so the simple least-squares solution equals the maximum likelihood estimator (MLE) of $\vec{\beta}$!

\section{Model Selection}
So solving the least-squares (or equivalently the MLE) problem gives an estimate of the parameter vector $\vec{\beta}$ for a given model $M_g$ that describes the given measured data best in a least-squares sense (note this is usually not outlier-robust!). This procedure can be repeated for any model $M_g$, and will result in a best-fit parameter vector $\vec{\beta}_{M_g}^\ast$ for each of these models, but the question remains: Which model should be taken, especially when the model selection is not obvious from external assumption or knowledge about the sample. In these cases a statistical model selection should be done. 

\subsection{$\chi^2$ model selection criterion}
The simplest model selection is based on the $\chi^2$ criterion, which is simply defined as the sum of squared deviations (sometimes also called residual sum of squares, RSS):
\begin{equation}\label{eq:chi2_rss}
  \chi^2(\vec{\beta},M_g)=\mbox{RSS}(\vec{\beta},M_g)=\sum\limits_{i=1}^n\left|\frac{g(\hat{\tau}_i;\vec{\beta})-\hat{g}_i}{\hat{\sigma}_i}\right|^2
\end{equation}
With this criterion, the decision rule is:
\begin{center}
  \bfseries Use the model that has the lowest $\chi^2(\vec{\beta},M_g)$, \\i.e. is closest to the measured data.
\end{center}
Unfortunately this simple model is not ideal, as a more complex model $M_g$ (i.e. with more parameters in $\vec{\beta}$ often also has a lower $\chi^2$-value, since the added model complexity allows to describe more of the noise on the data.

\subsection{Other model selection criteria}
To overcome these problems, other model selection criteria \cite{BURNHA2002} have been proposed, e.g. the Akaike's information criterion (AIC, \cite{AKAIKE1974,BURNHA2002}):
\begin{align}
  \mbox{AIC}(\vec{\beta},M_g)&=-2\ln\bigl[p(\hat{\vec{g}}|M_g,\vec{\beta})\bigr]+2k, \label{eq:aic}\\
  \mbox{AICc}(\vec{\beta},M_g)&=\mbox{AIC}(\vec{\beta},M_g)+\frac{2k\cdot(k+1)}{n-k-1}\ \ \ \text{(corrected for small $n$)} \label{eq:aicc}
\end{align}
or the Bayesian information criterion (BIC, also known as Schwarz's criterion, \cite{SCHWAR1978,BURNHA2002}):
\begin{equation}\label{eq:bic}
  \mbox{BIC}(\vec{\beta},M_g)=-2\ln\bigl[p(\hat{\vec{g}}|M_g,\vec{\beta})\bigr]+k\cdot\ln [n]
\end{equation}
For both criteria, the model selection rule is:
\begin{center}
  \bfseries Use the model that has the smallest (often most negative) value of AIC or BIC.
\end{center}
Since both criteria include the number of parameters $k$, they also obey (to some extent) the principle of parsimony (or Occam's razor), which states that one should use the model that best describes the data \underline{and} has the lowest number of parameters/is the simplest.

For any practical purposes, both criteria can be estimated also from the $\chi^2$, but only up to a fixed additive constant, which only depends on the dataset $(\hat{\tau}_i, \hat{g}_i, \hat{\sigma}_i)$ and therefore does not play a role for the selection process. This can be seen, if we write the log-likelihood, assuming Gaussian errors as follows:
\begin{equation}\label{eq:simple_loglikelihood}
  \ln\bigl[p(\hat{\vec{g}}|M_g,\vec{\beta})\bigr]=\underbrace{-\frac{n}{2}\cdot\ln(2\pi)-\sum\limits_i\ln\left[\hat{\sigma}_i\right]}_{=\text{const}}-\frac{1}{2}\cdot\sum\limits_{i=1}^n\frac{\left(g(\hat{\tau}_i;\vec{\beta})-\hat{g}_i\right)^2}{\hat{\sigma}_i^2}%=\\
  =\text{const}-\frac{\chi^2}{2}
\end{equation}
So we get:
\begin{equation}\label{eq:aic_bic_chi2}
  \mbox{AIC}(\vec{\beta},M_g)=\chi^2+2k\ \ \ \ \ \text{and}\ \ \ \ \ \mbox{BIC}(\vec{\beta},M_g)=\chi^2+k\cdot\ln [n]
\end{equation}
If all $\hat{\sigma}_i=\sigma$ are equal  (i.e. a non-weighted fit), then the estimations in \eqref{eq:simple_loglikelihood} change \cite{BURNHA2002}:
\begin{align}\label{eq:simple_loglikelihood_equalvar}
  \ln\bigl[p(\hat{\vec{g}}|M_g,\vec{\beta})\bigr]&=\-\frac{n}{2}\cdot\ln(2\pi)-\sum\limits_i\ln\left[\sigma\right]-\frac{1}{2}\cdot\sum\limits_{i=1}^n\left(\frac{g(\hat{\tau}_i;\vec{\beta})-\hat{g}_i}{\sigma}\right)^2=\\
  &=-\frac{n}{2}\cdot\ln(2\pi)-\frac{n}{2}\cdot\ln\left[\sigma^2\right]-\frac{1}{2\sigma^2}\cdot\underbrace{\sum\limits_{i=1}^n\left(g(\hat{\tau}_i;\vec{\beta})-\hat{g}_i\right)^2}_{=\chi^2}=\\
  &=\text{const}-\frac{n}{2}\cdot\ln\left[\frac{\chi^2}{n}\right]
\end{align}
Here, in the last step we used the estimator $\chi^2/n$ for the sample variance $\sigma^2$. Then the last term is also constant and the AIC and BIC become:
\begin{equation}\label{eq:aic_bic_chi2_equalsigma}
  \mbox{AIC}(\vec{\beta},M_g)=n\cdot\ln\left[\frac{\chi^2}{n}\right]+2k\ \ \ \ \ \text{and}\ \ \ \ \ \mbox{BIC}(\vec{\beta},M_g)=n\cdot\ln\left[\frac{\chi^2}{n}\right]+k\cdot\ln [n]
\end{equation}


\subsection{Bayesian model selection}
Another framework for model selection \cite{HE2012,GUO2012} is based in the Bayes theorem, thus the name ``Bayesian model selection''. The Bayes theorem can be used to calculate the probability $p(M_g|\hat{\vec{g}})$ for a specific model $M_g$, given a measurement $(\hat{\vec{\tau}}, \hat{\vec{g}}, \hat{\vec{\sigma}})$:
\begin{equation}\label{eq:bayes_theorem}
    p(M_g|\hat{\vec{g}})=\frac{p(\hat{\vec{g}}|M_g)\cdot p(M_g)}{p(\hat{\vec{g}})}.
\end{equation}
Here $p(M_g)$ is the prior (probability) of a model $M_g$, which allows to insert any prior/external information into the selection problem. The probability $p(\hat{\vec{g}})$ is mostly a normalization constant, as no statement about the absolute probability of a given dataset can easily be done. Finally $p(\hat{\vec{g}}|M_g)$ is the conditional probability of obtaining the measurement $\hat{\vec{g}}$, given the specified model $M_g$. As will be shown later. This can be calculated from the likelihood function \eqref{eq:fcs_likelihood_simple}. 

In the special case of model selection, we often do not want to put any prior information into the problem, as we cannot say which model is more likely, a priori. Therefore we will assume a flat prior $p(M_g)$ here. As said above, the probability of the data $p(\hat{\vec{g}})$ is treated as a normalization constant, thus we can simplify \eqref{eq:bayes_theorem} to:
\begin{equation}\label{eq:bayes_theorem_simplified}
    p(M_g|\hat{\vec{g}})\propto p(\hat{\vec{g}}|M_g).
\end{equation}
We will then calculate only the probability $p(\hat{\vec{g}}|M_g)$ for each model $M_g$. Then the decision rule is simple:
\begin{center}
  Choose the model $M_g$ with the highest (non-normalized) probability $p(\hat{\vec{g}}|M_g)$.
\end{center}
From the $p(\hat{\vec{g}}|M_g)$, it is also possible to calculate absolute model probabilities, by normalizing the $p(\hat{\vec{g}}|M_g)$ as follows:
\begin{equation}\label{eq:bayes_modelprob_normalized}
    p_\text{norm}(M_g|\hat{\vec{g}})= \frac{p(\hat{\vec{g}}|M_g)}{\sum\limits_i p(\hat{\vec{g}}|M_i)},\ \ \ \ \ \text{i.e.}\ \ \ \ \sum\limits_g p_\text{norm}(M_g|\hat{\vec{g}})=1.
\end{equation}

So finally we are left with the problem of calculating $p(\hat{\vec{g}}|M_g)$. Following  \cite{HE2012} this can be done from the likelihood function $p(\hat{\vec{g}}|M_g,\vec{\beta})$ by ``integrating out'' or ``marginalizing'' the parameter vector $\vec{\beta}$:
\begin{equation}\label{eq:bayes_propfromlikelihood}
    p(\hat{\vec{g}}|M_g,\vec{\beta})=\int\limits_{\vec{\beta}} p(\hat{\vec{g}}|M_g,\vec{\beta})\cdot p(\vec{\beta}|M_g)\ \mathrm{d}\vec{\beta}
\end{equation}
Here again the probability $p(\vec{\beta}|M_g)$ is the prior probability distribution of the parameters, which we also assume to be uniform, since we do not have any a priori information on the parameters. If the solution of this integral is not possible analytically -- which is often the case, as the fit functions are complex and the parameter space is high-dimensional -- it can be solved either numerically, with Monte Carlo integration, or in the Laplace approximation\cite{KASS1995}\footnote{The Laplace approximation is a method that allows to approximate the value of integrals of the form \[\int h(x)\cdot \exp\left(M\cdot f(x)\right)\;\mathrm{d}x,\] $M$ is a large number, $f(x)$ is a twice-differentiable function that has a strong peak at a position $x^\ast$ (e.g. a Gaussian distribution) and $h(x)$ is a smooth function. Then Laplace derived that \cite{Lauritzen_laplace}:
\[\int h(x)\cdot \exp\left(M\cdot f(x)\right)\;\mathrm{d}x \ \ \ \ \approx\ \ \ \  \mathrm{e}^{-M\cdot g(x^\ast)}\cdot h(x^\ast)\cdot\sqrt{\frac{2\pi}{M\cdot f''(x^\ast)}}\cdot\left[1+\mathcal{O}(1/M)\right]\] Here $f''(x)$ is the second derivative of the function $f(\cdot)$. A simple application of this approximation is Stirling's approximation formula for the $\Gamma(\cdot)$ function. }.
With Laplace's approximation and the assumption of Gaussian errors as in \eqref{eq:fcs_likelihood_simple} or \eqref{eq:fcs_likelihood_covmatrix} the marginalized probability of a given model $M_g$ is:
\begin{equation}\label{eq:bayes_marginprob_laplace}
    p(\hat{\vec{g}}|M_g,\vec{\beta})=(2\pi)^{k/2}\cdot\sqrt{\det(\mat{\Sigma}_\text{Bayes})}\cdot p(\hat{\vec{g}}|M_g,\vec{\beta}^\ast_\text{Bayes})\cdot p(\vec{\beta}^\ast_\text{Bayes}|M_g).
\end{equation}
Here again $p(\vec{\beta}^\ast_\text{Bayes}|M_g)$ is the prior probability of the parameters, at the ideal solution $\vec{\beta}^\ast_\text{Bayes}$ of the Bayesian fit problem (cf. also \eqref{eq:fcs_fit_problem_maxlikelihood}):
\begin{equation}\label{eq:fcs_fit_problem_maxBayeslikelihood}
  \vec{\beta}_{Bayes}^\ast=\argmax\limits_{\vec{\beta}}p(\hat{\vec{g}}|M_g,\vec{\beta})\cdot p(\vec{\beta}|M_g)=\argmax\limits_{\vec{\beta}}\ln\left[p(\hat{\vec{g}}|M_g,\vec{\beta})\cdot p(\vec{\beta}|M_g)\right],
\end{equation}
and 
\begin{equation}\label{eq:fcs_fit_problem_invbayeshessian}
  \mat{\Sigma}_\text{Bayes}=\left\{-\vec{\nabla}\vec{\nabla}\ln\left[p(\hat{\vec{g}}|M_g,\vec{\beta})\cdot p(\vec{\beta}|M_g)\right]_{\vec{\beta}=\vec{\beta}_{Bayes}^\ast}\right\}^{-1}
\end{equation}
is the inverse Hessian matrix, or variance-covariance matrix of the Bayesian fit problem \eqref{eq:fcs_fit_problem_maxBayeslikelihood}. As already done at several occasions above, we can now again take the priors to be flat. Then the Bayes fit problem \eqref{eq:fcs_fit_problem_maxBayeslikelihood} equals the simple maximum likelihood fit \eqref{eq:fcs_fit_problem_maxlikelihood} or (equivalently) the least-squares fit problem \eqref{eq:fcs_fit_problem}. Then $\vec{\beta}_{Bayes}^\ast$ is given by the solution to those problems and $\mat{\Sigma}_\text{Bayes}$ can be estimated by the variance-covariance matrices from section~\ref{sec:varcovmatrix}, i.e. equation~\eqref{eq:varcov_noerrors}. In this approximation, 
equation \eqref{eq:bayes_marginprob_laplace} then simplifies to:
\begin{multline}\label{eq:bayes_marginprob_laplace_flatprior}
    p(\hat{\vec{g}}|M_g,\vec{\beta})\approx(2\pi)^{k/2}\cdot\sqrt{\det(\mat{\Sigma})}\cdot \exp\left\{-\frac{n\cdot\ln(2\pi)}{2}-\sum\limits_{i=1}^n\ln(\hat{\sigma}_i)-\frac{1}{2}\cdot\sum\limits_{i=1}^n\left(\frac{g(\hat{\tau}_i;\vec{\beta}^\ast)-\hat{g}_i}{\hat{\sigma}_i}\right)^2\right\}=\\
    =\exp\left\{\frac{k}{2}\cdot\ln(2\pi)+\frac{1}{2}\cdot\ln\left[\det(\mat{\Sigma})\right] -\frac{n\cdot\ln(2\pi)}{2}-\sum\limits_{i=1}^n\ln(\hat{\sigma}_i)-\frac{1}{2}\cdot\sum\limits_{i=1}^n\left(\frac{g(\hat{\tau}_i;\vec{\beta}^\ast)-\hat{g}_i}{\hat{\sigma}_i}\right)^2\right\}=\\
    =\exp\left\{\frac{k}{2}\cdot\ln(2\pi)+\frac{1}{2}\cdot\ln\left[\det(\mat{\Sigma})\right] -\sum\limits_{i=1}^n\ln(\hat{\sigma}_i)-\frac{1}{2}\cdot\sum\limits_{i=1}^n\left(\frac{g(\hat{\tau}_i;\vec{\beta}^\ast)-\hat{g}_i}{\hat{\sigma}_i}\right)^2\right\}\cdot\exp\left\{-\frac{n\cdot\ln(2\pi)}{2}\right\}.
\end{multline}



\newpage
\bibliographystyle{plainnat}
\bibliography{bayesian_fit}
\end{document}
