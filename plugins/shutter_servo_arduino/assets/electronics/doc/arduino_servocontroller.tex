%% Artikelvorlage unter Verwendung der Artikelklasse des KOMA-Script %%
%% Basierend auf einer TeXNicCenter-Vorlage von Mark M�ller          %%
%%%%%%%%%%%%%%%%%%%%%%%%%%%%%%%%%%%%%%%%%%%%%%%%%%%%%%%%%%%%%%%%%%%%%%%

% W�hlen Sie die Optionen aus, indem Sie % vor der Option entfernen  
% Dokumentation des KOMA-Script-Packets: scrguide

%%%%%%%%%%%%%%%%%%%%%%%%%%%%%%%%%%%%%%%%%%%%%%%%%%%%%%%%%%%%%%%%%%%%%%%
%% Optionen zum Layout des Artikels                                  %%
%%%%%%%%%%%%%%%%%%%%%%%%%%%%%%%%%%%%%%%%%%%%%%%%%%%%%%%%%%%%%%%%%%%%%%%
\documentclass[%
%a5paper,							% alle weiteren Papierformat einstellbar
%landscape,						% Querformat
10pt,								% Schriftgr��e (12pt, 11pt (Standard))
%BCOR1cm,							% Bindekorrektur, bspw. 1 cm
%DIVcalc,							% f�hrt die Satzspiegelberechnung neu aus
%											  s. scrguide 2.4
%twoside,							% Doppelseiten
%twocolumn,						% zweispaltiger Satz
%halfparskip*,				% Absatzformatierung s. scrguide 3.1
%headsepline,					% Trennline zum Seitenkopf	
%footsepline,					% Trennline zum Seitenfu�
%titlepage,						% Titelei auf eigener Seite
%normalheadings,			% �berschriften etwas kleiner (smallheadings)
%idxtotoc,						% Index im Inhaltsverzeichnis
%liststotoc,					% Abb.- und Tab.verzeichnis im Inhalt
%bibtotoc,						% Literaturverzeichnis im Inhalt
%abstracton,					% �berschrift �ber der Zusammenfassung an	
%leqno,   						% Nummerierung von Gleichungen links
%fleqn,								% Ausgabe von Gleichungen linksb�ndig
%draft								% �berlangen Zeilen in Ausgabe gekennzeichnet
]
{article}


% Seiten-Gr��e einstellen ist mit dem Geometry-Package viel einfacher
%  a4paper                DIN A4-Papier
%  text={width,height}    gibt H�he und Breite des Text-Bereiches an
%  centering              zentriert den Textbereich auf der Seite
%  landscape|portrait     Quer-/Hochformat
\usepackage[a4paper,text={160mm,255mm},centering,headsep=5mm,footskip=10mm]{geometry}

%\pagestyle{empty}		% keine Kopf und Fu�zeile (k. Seitenzahl)
%\pagestyle{headings}	% lebender Kolumnentitel  


%% Fonts f�r pdfLaTeX, falls keine cm-super-Fonts installiert %%%%%%%%
%\ifpdf
	%\usepackage{ae}        % Benutzen Sie nur
	%\usepackage{zefonts}  	% eines dieser Pakete
%\else
	%%Normales LaTeX - keine speziellen Fontpackages notwendig
%\fi

%% optischer Randausgleich, falls pdflatex verwandt %%%%%%%%%%%%%%%%%%%
\usepackage[activate]{pdfcprot}				

%% Packages f�r Grafiken & Abbildungen %%%%%%%%%%%%%%%%%%%%%%
\usepackage{graphicx} %%Grafiken in pdfLaTeX
%\usepackage[hang]{subfigure} %%Mehrere Teilabbildungen in einer Abbildung
%\usepackage{pst-all} %%PSTricks - nicht verwendbar mit pdfLaTeX

%% deutsche Anpassung %%%%%%%%%%%%%%%%%%%%%%%%%%%%%%%%%%%%%%%%%%%%%%%%%
\usepackage[english]{babel}		
\usepackage[T1]{fontenc}							
\usepackage[latin1]{inputenc}		

\usepackage{color}
\usepackage{hyperref}
\definecolor{darkblue}{rgb}{0,0,0.3}
\definecolor{darkgreen}{rgb}{0,.3,0}
\definecolor{darkred}{rgb}{.3,0,0}
\hypersetup{pdftex=true,citecolor=black,colorlinks=true,breaklinks=true,linkcolor=black,menucolor=black,pagecolor=black,urlcolor=black}

\usepackage{amsmath}
\usepackage{longtable}
\usepackage{ifsym}
\usepackage{float}


\begin{document}
%% Dateiendungen f�r Grafiken %%%%%%%%%%%%%%%%%%%%%%%%%%%%%%%
%% ==> Sie k�nnen hiermit die Dateiendung einer Grafik weglassen.
%% ==> Aus "\includegraphics{titel.eps}" wird "\includegraphics{titel}".
%% ==> Wenn Sie nunmehr 2 inhaltsgleiche Grafiken "titel.eps" und
%% ==> "titel.pdf" erstellen, wird jeweils nur die Grafik eingebunden,
%% ==> die von ihrem Compiler verarbeitet werden kann.
%% ==> pdfLaTeX benutzt "titel.pdf". LaTeX benutzt "titel.eps".
\ifpdf
	\DeclareGraphicsExtensions{.pdf,.jpg,.png}
\else
	\DeclareGraphicsExtensions{.eps}
\fi

\pagestyle{empty} %%Keine Kopf-/Fusszeilen auf den ersten Seiten.


%%%%%%%%%%%%%%%%%%%%%%%%%%%%%%%%%%%%%%%%%%%%%%%%%%%%%%%%%%%%%%%%%%%%%%%
%% Ihr Artikel                                                       %%
%%%%%%%%%%%%%%%%%%%%%%%%%%%%%%%%%%%%%%%%%%%%%%%%%%%%%%%%%%%%%%%%%%%%%%%

%% eigene Titelseitengestaltung %%%%%%%%%%%%%%%%%%%%%%%%%%%%%%%%%%%%%%%    
%\begin{titlepage}
%Einsetzen der TXC Vorlage "Deckblatt" m�glich
%\end{titlepage}

%% Angaben zur Standardformatierung des Titels %%%%%%%%%%%%%%%%%%%%%%%%
%\titlehead{Titelkopf }
%\subject{Typisierung}
\title{Arduino Servo Controller}
\author{Jan Krieger <j.krieger@dkfz.de> / <jan@jkrieger.de>\footnote{German Cancer Research Center (DKFZ), Department: Biophysics of Macromolecules (Prof. J. Langowski), Im Neuenheimer Feld 580, D-69120 Heidelberg, Germany, tel: +49-6221/42-3395}}
%\and{Der Name des Co-Autoren}
%\thanks{Elisabeth Brama, now @ University of Sussex, Brighton}			% entspr. \footnote im Flie�text
%\date{}							% falls anderes, als das aktuelle gew�nscht
%\publishers{Herausgeber}

%% Widmungsseite %%%%%%%%%%%%%%%%%%%%%%%%%%%%%%%%%%%%%%%%%%%%%%%%%%%%%%
%\dedication{Widmung}

\maketitle 						% Titelei wird erzeugt

%% Zusammenfassung nach Titel, vor Inhaltsverzeichnis %%%%%%%%%%%%%%%%%
%\begin{abstract}
% F�r eine kurze Zusammenfassung des folgenden Artikels.
% F�r die �berschrift s. \documentclass[abstracton].
%\end{abstract}

%% Erzeugung von Verzeichnissen %%%%%%%%%%%%%%%%%%%%%%%%%%%%%%%%%%%%%%%
\tableofcontents			% Inhaltsverzeichnis
%\listoftables				% Tabellenverzeichnis
%\listoffigures				% Abbildungsverzeichnis


%% Der Text %%%%%%%%%%%%%%%%%%%%%%%%%%%%%%%%%%%%%%%%%%%%%%%%%%%%%%%%%%%
\newpage
\section{Introduction}
\label{sec:Introduction}

This circuit controls up to four servos connected to the I/O ports of an Arduino Nano. A USB-to-serial converter allows a host PC to control the servos. The circuit is optimized to �se the servos as laser shutters, so each servo may only reside in one of two possible states (0 or 1).

\section{USB Interface \& Command reference}
\label{sec:USBInterfaceCommandReference}
\subsection{USB Interface}
\label{sec:USBInterface}
As mentioned before the control box contains a USB to serial converter. Thus the communication with a PC may be done by simply sending text commands over a terminal (e.g. \url{http://sites.google.com/site/braypp/terminal} for windows). Several commands are defined that allow to control the laser and the remote control box. Each command starts with a single character identifying it and the additional characters that give parameters. Commands with additional parameters require a closing line feed character (\texttt{0x0A}, or \texttt{$\backslash$n} in C/C++). Any characters which are not recognized as commands will be ignored, so the interface usually does not stall. If a command features a return value two line feed characters (\texttt{0x0A}, or \texttt{$\backslash$n} in C/C++) end the return value.

The interface has a configurable baud rate (fixed to 9600 baud), uses 8 data bits, no parity and one stop bit. The interface is implemented, using a serial to USB converter chip from FTDI. These chips register to the operating system as a virtual serial port, if an appropriate driver is installed. There are drivers for many Windows variants, Linus and MacOS 8,9 and X. Usually a Linux kernel will recognize the interface out of the box. The drivers may be downloaded and installed from \url{http://www.ftdichip.com/Drivers/VCP.htm}.



\subsection{Command Reference}
\label{sec:CommandReference}
This section will give an overview over the available commands together with examples (identifiers in brackets \texttt{<...>} have to be replaced by the described data, a value in brackets like \texttt{<0xD4>} stands for the binary representation of the value, <LF> stands for a line feed character, <CR> for carriage return). Note that commands are generally case-sensitive!
\begin{center}
	\begin{longtable}{|p{60mm}|p{90mm}|}\hline
	\textit{command} & \textit{description}\\\hline
	\endhead
	\hline
	\endfoot
	\hline
	\endlastfoot

	\multicolumn{2}{l}{\textbf{Status Report \& Configuration}}\\\hline
	\texttt{?} & identifies the remote control box and sends back copyright and version information (multi-line output ended by two consecutive \texttt{<LF>} characters. Example output: \begin{verbatim}Servo Controller<LF>
  (c) 07.2010 by Jan Krieger (DKFZ)<LF>
  j.krieger@dkfz.de --  jan@jkrieger.de<LF><LF>\end{verbatim}\\\hline
	\texttt{V} & return version information. Example output: \begin{verbatim}1.3.319.1735 (05.2010)<LF>\end{verbatim}\\\hline

	\multicolumn{2}{l}{\textbf{Servo Control \& Status}}\\\hline
	\texttt{s<SERVO><STATE>} & set the given servo (1..4) to the given state (0,1) \\\hline
	\texttt{q<SERVO>} & returns the state (0,1) of the given servo (1..4) \\\hline

	\end{longtable}
\end{center}


\section{Technical Realization}
\label{sec:Realization}
\subsection{Hardware Overview (Electronics)}
\label{sec:HardwareOverview}


\subsection{Implementation Details and Hints}
\label{sec:ImplementationDetailsAndHints}
\begin{itemize}
  \item The FTDI USB to serial converter has areprogrammed device name 
		\begin{center}
			\textit{***}
		\end{center}
		Still you can use the standard FTDI virtual COM port driver for this device: 
		\begin{center}
			\url{http://www.ftdichip.com/FTDrivers.htm}
		\end{center}

	\item The FTDI USB to serial converter has to be programmed to be self-powered. Otherwise the serial connection does not work on Linux (on Windows system the problem does not seem to exist).
	\item You can bind a box to a spoecific serial port. On \textbf{Windows} this can be done in the driver settings. On \textbf{Linux} this may be achieved using the \texttt{udev} hardware enumeration system:
	
\begin{enumerate}
	\item create a udev rules file with a name like e.g.: \texttt{/etc/udev/rules.d/99-usbboxes.rules} 
	\item add a line like this:

{\footnotesize\begin{verbatim}
  #LASERBox
  KERNEL=="ttyUSB*", SUBSYSTEMS=="usb", ATTRS{product}=="B040SERVO", SYMLINK+="ttyUSB_B040SERVO" 
\end{verbatim}}

    which will map the ttyUSB device with the product name B040SERVO to the device file \texttt{/dev/ttyUSB\_B040SERVO} in addition to its default file \texttt{/dev/ttyUSB}\textit{x}. Instead of the product name you can also use any other USB device property, as reported by

{\footnotesize\begin{verbatim}
  lsusb -v -s <bus>:<devnum>
\end{verbatim}}

    Possible filters are e.g.:

{\footnotesize\begin{verbatim}
   ATTRS{serial}=="A800dOBl", 
   ATTRS{idVendor}=="1a72", 
   ATTRS{idProduct}=="1007", 
   ATTR{vendor}=="0x149a", 
   ATTR{device}=="0x0005"
\end{verbatim}}

    \item After saving the rules file you will have to tell udev that there are new/changed rules by executing

{\footnotesize\begin{verbatim}
  /etc/init.d/udev restart
\end{verbatim}}

    The rules will be executed when the new device is connected to the computer. So if it's already connected: pull and reconnect the plug.

\end{enumerate}
\end{itemize}






\newpage
\section{Literature \& Datasheets}
\label{sec:literature_datasheets}
\begin{itemize}
  \item FT232RL: \url{http://www.ftdichip.com/Documents/DataSheets/DS_FT232R.pdf}
\end{itemize}
\end{document}

